\documentclass[11pt, letterpaper, onecolumn]{article}
 
\usepackage[english]{babel}
\usepackage{soul}
\usepackage{mathtools}
\usepackage[utf8]{inputenc}
\usepackage{graphicx}
\usepackage{float}
\usepackage[german=quotes]{csquotes}
\usepackage{hyperref}
\usepackage{fancyhdr}
\usepackage{gensymb}
\usepackage{units}
\usepackage{hhline}
\usepackage{color}
\usepackage{titling}
\usepackage[normalem]{ulem}
\usepackage[margin=2.5cm]{geometry}
\usepackage{amsmath}
\usepackage{amssymb}
\usepackage{amsfonts}
\usepackage{pgfplots}
\usepackage{array}
\usepackage{makecell}
\usepackage{subfigure}
\usepackage{lipsum}
\usepackage{url}
\usepackage{relsize}

\newgeometry{a4paper, left=20mm, right=20mm, top=30mm, bottom=30mm}
\definecolor{pantone294}{cmyk}{1,0.6,0,0.2}
\setlength{\columnsep}{6mm} 

\title{Project 1: QM point particle in external potential} 
\author{Florian Telleis / Florian Hollants / Mickey Wilke}
\date{\today}

\pagestyle{fancy}
\lfoot{Humboldt-Universität zu Berlin}
\rfoot{Project 1.1} 



\begin{document}
	
	\newgeometry{left=14mm, right=13.5mm, top=13.5mm, bottom=30mm}
	\begin{titlepage}
		\thispagestyle{empty}
		\begin{figure}
			
		\end{figure}
		\vspace*{-43mm}\hspace{-6mm}\textbf{\textcolor{pantone294}{\large{Mathematisch-Naturwissenschaftliche Fakultät}}}\\\\\\\\\\
		\textcolor{pantone294}{Institut für Physik}\\
		\vspace{30mm}
		\begin{center}
			\textcolor{pantone294}{\huge{Computational Physics II}}\\\vspace*{7mm}
			\textcolor{pantone294}{\huge{\textbf{\thetitle}}}\\\vspace*{10mm}
			\textcolor{pantone294}{\theauthor}\\\vspace*{10mm}
			\textcolor{pantone294}{\thedate}\\\vspace*{20mm}
			\begin{tabular}{ll}
				\textbf{Work Group:} & Albert Einstein	 \\ \\
				\textbf{Students:} & Florian Telleis (612716) \\
									& Florian Hollants (------)\\
									& Mickey  Wilke (------)\\ \\
				\textbf{Submitted:} & --.11.2024 \\ \\				
			\end{tabular}
		\end{center}
	\end{titlepage}
	\makeatother
	\restoregeometry
		
		\newpage
	
	
	
	
	\begin{center}
	\Large{Abstract}
	\end{center}
	k\\
	\vspace{1cm}
	
	
	
	\tableofcontents
	
	
	
	
	
	\vspace{50cm}
	
	\section{Introduction}
	









%	\hyperref[Quellen]{$^{[1]}$}	



	
%		\begin{figure} [h] 
%	\begin{center}
%	\subfigure[CGM]{\includegraphics[width=0.45\textwidth]{799px-Constant_current.svg.png}}
%    \subfigure[CHM]{\includegraphics[width=0.45\textwidth]{799px-Constant_height.svg.png}}
%\caption{Verschiedene Modi für das RTM Abrastern, entnommen aus \hyperref[Quellen]{[3]}}
%	\end{center}
%	\end{figure}


	

	
	%	\begin{figure} [h] 
%	\begin{center}
%	\includegraphics[width=8.8cm]{"QBER(R).jpg"}
%	\caption{QBER($R_{det}$) for three different photon sources and different attenuations; the data points are linearly connected for better visibility}
%	\end{center}
%	\end{figure}
	
	
	

%	\hyperref[Quellen]{$^{[1]}$}	

	
	
\newpage
	

	
\section{Sources and Literature} \label{sources}

		$[1]$ \textit{Titel} - Autor; (Vers.) Datum
\vspace{0.4cm}
		 \\
		$[2]$ Internetseite: \textit{Titel} \\ \url{Link} \\- zuletzt besucht am Datum um Zeit
%		\vspace{0.4cm}
%		 \\
		
		
%		$[1]$ \textit{Titel} - Autor; (Vers.) Datum
%\vspace{0.4cm}
%		 \\
%		$[2]$ Internetseite: \textit{Titel} \\ \url{Link} \\- zuletzt besucht am Datum um Zeit
%		\vspace{0.4cm}
%		 \\

	
	

	
\section{Appendix} \label{appendix}
	
	
	





	
\end{document}
